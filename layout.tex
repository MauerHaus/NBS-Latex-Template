\documentclass[a4paper,11pt,DIV=calc,german]{scrartcl}
%\documentclass[DIV=calc,german]{scrartcl}
\usepackage[a4paper, left=3cm, right=3cm, top=2.5cm, bottom=2.5cm]{geometry}
%\usepackage[default]{sourcesanspro}
\usepackage[T1]{fontenc}
\usepackage[utf8]{inputenc}
\usepackage[german]{babel}
\usepackage{enumitem}
\usepackage[svgnames,table,hyperref]{xcolor}
\usepackage{listings}
\usepackage[breaklinks=true]{hyperref}
\usepackage{graphicx}
\usepackage{wrapfig}
\usepackage{subcaption}
\usepackage{amsmath}
\usepackage{multicol}
\usepackage{float}
\usepackage{threeparttable}
\usepackage{color}															 
% für Farben im allgemeinen
\usepackage{colortbl}	
\usepackage{eurosym}
%\usepackage{subfigure}
%\usepackage{cite}
\usepackage[printonlyused]{acronym}
\usepackage{sidecap}
%\usepackage{SIunits}
\usepackage{tikz}
%\usepackage{ulem}
\usepackage{fancybox}
\usepackage{amssymb}
\usepackage{wasysym}
\usepackage{mathpazo}
\usepackage{lmodern}
\usepackage{threeparttable}
%\usepackage{slashbox}
\usepackage{array}
\usepackage{multirow}
\usepackage{rotating}
\usepackage{adjustbox}
\usepackage{multirow}
\usepackage{colortbl}
\usepackage{acronym}
\usepackage{csquotes}


\usepackage[
backend=biber,
style=apa, %Zitationsstil
maxcitenames=3,	% mindestens 3 Namen ausgeben bevor et. al. kommt
maxbibnames=999,
date=iso,
seconds=true, %werden nicht verwendet, so werden aber Warnungen unterdrückt.
urldate=iso,
dashed=false,
dateera=astronomical,
autocite=inline,
useprefix=true, % 'von' im Namen beachten (beim Anzeigen)
mincrossrefs = 1
]{biblatex}%iso dateformat für YYYY-MM-DD

\numberwithin{figure}{section} %Zähler der Figures nach Kapitel, Weglassen dieser Zeile: dokumentenweite Zählung
\numberwithin{equation}{section} %Zählweise der Gleichungen
\usepackage{stmaryrd}
\addbibresource{Abschlussarbeit.bib}


\usetikzlibrary{calc,trees,positioning,arrows,chains,shapes.geometric,%
    decorations.pathreplacing,decorations.pathmorphing,shapes,%
    matrix,shapes.symbols}

\tikzset{
>=stealth',
  punktchain/.style={
    rectangle, 
    rounded corners, 
    % fill=black!10,
    draw=black, very thick,
    text width=10em, 
    minimum height=3em, 
    text centered, 
    on chain},
  line/.style={draw, thick, <-},
  element/.style={
    tape,
    top color=white,
    bottom color=blue!50!black!60!,
    minimum width=8em,
    draw=blue!40!black!90, very thick,
    text width=10em, 
    minimum height=3.5em, 
    text centered, 
    on chain},
  every join/.style={->, thick,shorten >=1pt},
  decoration={brace},
  tuborg/.style={decorate},
  tubnode/.style={midway, right=2pt},
}

\lstloadlanguages{[LaTeX]TeX}
\lstset{%
  basicstyle=\footnotesize\ttfamily,
  extendedchars=true,
  numbers=left,
  numberstyle=\tiny,
  stepnumber=1,
  numbersep=6pt,
  frame=single,
  frameround=tttt,
  captionpos=b,
  breaklines=true,
  aboveskip=2ex,
  belowskip=1ex,
  backgroundcolor=\color{LightYellow},
  commentstyle=\color{DarkRed},
  stringstyle=\ttfamily,
  showstringspaces=false,
  language={[LaTeX]TeX}
}
\hypersetup{%
  colorlinks=true,
  citecolor=black,
  linkcolor=black,
  %bookmarks=true,
  bookmarksnumbered=true,
  pdfauthor={\studentname},
  pdfcreator={Latex},
  pdfkeywords={Abschlussarbeit},
  urlcolor=black,
  plainpages=false,
  breaklinks=true,
  pdftitle={Abschlussarbeit}
}
\setlist[description]{leftmargin=1cm, style=sameline}

\newcommand{\subsubsubsection}[1]{\paragraph{#1}\mbox{}\\}


% Setze die Tiefe des Inhaltsverzeichnis auf 4 Ebenen
\makeatletter
%same as \subsubsection but level 4
\renewcommand\paragraph{\@startsection{paragraph}{4}{\z@}%
                                     {-3.25ex\@plus -1ex \@minus -.2ex}%
                                     {1.5ex \@plus .2ex}%
                                     {\normalfont\normalsize\bfseries}}
% number \paragraph
\setcounter{secnumdepth}{4}
\makeatother
%-----------------------------------

% Mehrere Fussnoten nacheinander mit Komma separiert
\usepackage[hang,multiple]{footmisc}
\setlength{\footnotemargin}{1em}

% Verhindert, dass nur eine Zeile auf der nächsten Seite steht
\setlength{\marginparwidth}{2cm}
\usepackage[all]{nowidow}
\urlstyle{same}%gleiche Schriftart für Link und Text
\hypersetup{colorlinks=true, breaklinks=true, linkcolor=darkblack, citecolor=darkblack, menucolor=darkblack, urlcolor=darkblack, linktoc=all, bookmarksnumbered=false, pdfpagemode=UseOutlines, pdftoolbar=true}
